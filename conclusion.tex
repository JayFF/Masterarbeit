\chapter{Summary \& outlook}

The goal of this thesis was to establish a frabrication pipeline for narrow linewidth \tmdg monolayers, encapsulated in \hbng and embedded in a gate-structure. This was accomplished by combining the dry transfer technique of ``hot pick-up and stamping'' \cite{pizzocchero_hot_2016, tien_study_2016} with contact lithography and mechanical exfoliation. \textsc{Tmd} monolayers were produced by mechanically exfoliating material from a bulk crystal with adhesive tape. To obtain narrow linewidths they were encasulated in \hbng flakes by picking up \hbng and \tmdg flakes one by one with a \textsc{ppc}-coated \pdms-stamp and dropped in contact with gold electrodes. To verify the function and determine the maximum gate voltage the leaking current through the samples was measured using a lock-in amplifyer.

A combined gate-tunable mono- and bilayer sample of tungsten diselenide was analyzed by means of confocal laser spectroscopy. Both reflection and photoluminescence spectra were taken while sweeping the gate voltage to determine intrinsic doping and observe the transition between neutral and negative charge carrier density.

The observed spectral linewidth of 6--7 meV for the neutral exciton peak is a significant improvement over previous samples, especially because these values can be observed on most of the sample. The intrinsic homogeneous linewidth is believed to be below 2 meV, so there is still room for improvement.

The good linewidth-related resolution allowed the study of different spectral features while tuning the electron density as well as applying a magnetic field. As a result the transition between a neutral regime and a negatively charged regime could be observed, in which the neutral exciton peak vanishes along with peaks to the red, that are associated with momentum-indirect excitons. Instead, the trion double feature, that is split by exchange interaction appears 30 meV to the red of the exciton, along with a new set of phonon sidebands, that could be charged counterparts of the peaks in the neutral spectrum. The first phonon sideband, associated with a momentum-indirect exciton with an electron in the $Q$-valley, is energetically close to the trion. Therefore its behavior upon changing the gate voltage was of particular interest. The exchange splitting upon the transition to the charged regime as well as a change in the $g$-factor show, that this peak is different from the trion. This 

T

The same procedures can also be performed on other \tmds\!---for example the closely related \ws\!. During the work for this thesis, a gate-tunable sample of \ws was already produced and can be analyzed in the future.

But there are also possible advancements in the fabrication process and sample engineering. Uncontrolled thickness of the bottom \hbng flake as well as the 50-90 nm \sio dielectric mean, that in order to create significant electric field strengths, high voltages close to breakdown have to be applied to observe the transition between all relevant regimes. Because of negative intrinsic doping only the neutral and negative regime could be resolved in this thesis. However, there are viable options to navigate this limitation. One option is to integrate top and backgate into the heterostructure itself. Using graphene beneath the bottom \hbng flake or above the top, the distance between the gates could be substantially lowered. The result would be a higher field strength close to the sample. The other option would be to retain the substrate as backgate but to discard \sio as a dielectric. Using techniques such as atomic layer deposition of materials like aluminum oxide, a strong but thin dielectric can be fabricated. This would raise the possible field strength with less complexity of the van-der-Waals heterostructure.

But even without changing the fabrication process, the present techniques could be used to build new and interesting samples. Stacking different \tmdg monolayers on top of each other to form gate-tunable heterobilayers could yield a better understanding of features like second-harmonic generation and layer-indirect excitons. A challenge would be to align the crystal axis of the precursor monolayers, but this could also be an opportunity to study moiré effects in more detail, that originate from a lattice mismatch between the two layers.
Finally, heterostructures as well as single layer samples could be embedded in an optical micro-cavity while also being fully gate tunable. This could help the understanding of exciton-polaritons and other phenomena, connected to cavity physics of \tmds.