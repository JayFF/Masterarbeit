\chapter{Summary \& outlook}

The goal of this thesis was to establish a frabrication pipeline for narrow linewidth \tmdg monolayers, encapsulated in \hbn, embedded in a gate-structure. This was accomplished by combining the dry transfer technique of ``hot pick-up and stamping'' with contact lithography and mechanical exfoliation.

\textsc{Tmd} monolayers were produced by mechanically exfoliating material from a bulk crystal with adhesive tape. To obtain narrow linewidths they were encasulated in \hbng flakes by picking up \hbng and \tmdg flakes one by one with a \textsc{ppc}-coated \pdms-stamp and dropped in contact with lithographically written gold wires. To verify the function and determine the maximum gate voltage the leaking current through the samples was measured using a lock-in amplifyer.

A combined gate-tunable mono- and bilayer sample of tungsten diselenide was analyzed by means of confocal laser spectroscopy. Both reflection and photoluminescence spectra were taken while speeping the gate voltage to determine intrinsic doping and observe the transition between neutral and negative charge carrier density. To gain insight about the origin of the different spectral features, these sweeps were repeated for high positive and negative magnetic fields. The observed $g$-factors were used to distinguish between the negatively charged trion state and a not yet fully understood strong feature in the neutral spectrum. 

The $g$-factors of previously unidentified peaks could help to verify our model, that identifies them as phonon replica of momentum-indirect excitons. However, this will have to be completed by other measurements including life-time measurements, power dependence but most of all theoretical predictions of $g$-factors. 