\chapter{Summary \& outlook}

The goal of this thesis was to establish a frabrication pipeline for narrow linewidth \tmdg monolayers, encapsulated in \hbn, embedded in a gate-structure. This was accomplished by combining the dry transfer technique of ``hot pick-up and stamping'' with contact lithography and mechanical exfoliation.

\textsc{Tmd} monolayers were produced by mechanically exfoliating material from a bulk crystal with adhesive tape. To obtain narrow linewidths they were encasulated in \hbng flakes by picking up \hbng and \tmdg flakes one by one with a \textsc{ppc}-coated \pdms-stamp and dropped in contact with lithographically written gold wires. To verify the function and determine the maximum gate voltage the leaking current through the samples was measured using a lock-in amplifyer.

A combined gate-tunable mono- and bilayer sample of tungsten diselenide was analyzed by means of confocal laser spectroscopy. Both reflection and photoluminescence spectra were taken while speeping the gate voltage to determine intrinsic doping and observe the transition between neutral and negative charge carrier density. To gain insight about the origin of the different spectral features, these sweeps were repeated for high positive and negative magnetic fields. The observed $g$-factors were used to distinguish between the negatively charged trion state and a not yet fully understood strong feature in the neutral spectrum. 

The $g$-factors of previously unidentified peaks could help to verify our model, that identifies them as phonon replica of momentum-indirect excitons. However, this will have to be completed by other measurements including life-time measurements, power dependence but most of all theoretical predictions of $g$-factors. The same procedures have to be performed on other \tmds\!. Most of all the closely related \ws\!. During the work for this thesis, a sample of \ws with an optimal gate-tunability was already produced and can be analyzed in the future.

But there are also possible advancements in the fabrication process and sample engineering. Uncontrolled thickness of the bottom \hbng flake as well as the 50-90 nm \sio dielectric mean, that in order to create significant electric field strengths, high voltages close to the breakdown have to be applied to observe the transistion between all relevant regimes. Because of negative intrinsic doping only the neutral and negative regime could be resolved in this thesis. However, there are viable options to navigate this limitation. One option is to integrate top and backgate into the heterostructure itself. Using graphene beneath the bottom \hbng flake or above the top the distance between the gates could be substancially lowered. The result would be a much higher field strength close to the monolayer. The other option would be to still use the substrate as backgate but to discard \sio as a dielectric. Using techniques such as atomic layer deposition of materials like aluminum oxide, a similar effect could be achieved, without raising the complexity of the van-der-Waals heterostructure.

But it should not be forgotten, that even the present samples and samples fabricated in the same way still hold a lot of information, that can be retrieved, now that the linewidth of spectral lines poses a much weaker limitation. 