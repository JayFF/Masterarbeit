\documentclass{scrbook}
\usepackage[ngerman, main=english]{babel}
\usepackage{fontspec}
\usepackage{amsmath}
\usepackage{csquotes}
\usepackage{amssymb}
\usepackage{float}
\usepackage{hyperref}
\usepackage{cleveref}
\usepackage{nicefrac}
\usepackage[toc, page]{appendix}

\usepackage{mathpazo}
%\setsansfont{[futura_medium_bt.ttf]}
%\setmainfont[
% BoldFont={[Futura_Bold_font.ttf]}, 
% ItalicFont={Futura_Light_Italic_font.ttf},
% BoldItalicFont={[Futura_Bold_Italic_font.ttf]}
% ]{[futura_light_bt.ttf]}
\setmainfont[Numbers=OldStyle]{Linux Libertine O}
\setkomafont{sectioning}{\scshape}
\setkomafont{title}{\scshape}
%\setkomafont{section}{\rmfamily}

\usepackage{graphicx}
\usepackage{caption}
\usepackage{subcaption}
\usepackage[backend=biber,style=phys,sorting=none,natbib=true]{biblatex}
\addbibresource{references.bib}
\usepackage{remreset}

\renewcommand{\autodot}{}

\title{Fancy Masterarbeit of Death}
\author{Jonathan Förste}

\begin{document}
\maketitle
\tableofcontents
\chapter{Introduction}
\chapter{Physical Properties of Transision Metal Dichalcogenide Monolayers}
	\section{Crystal structure and symmetries}
	\section{Electronic bandstructure and spin orbit coupling}
	\section{Excitons in \textsc{tmd}s}
		\subsection{Binding energy}
		\subsection{Phonons and dark states}
		\subsection{Trions}
	\section{The Valley Zeeman Effect}
	\section{Bilayer WSe$_2$}
\chapter{Fabrication of field effect structures}
	\section{Mechanical exoliation}
		\subsection{Layer number}
	\section{Hexagonal boron nitide}
	\section{Electrode fabrication}
		\subsection{UV lithography}
		\subsection{Backgate and contacting}
	\section{Hot pickup and stamping}
	\section{Annealing}
\chapter{Experimental methods and results}
	\section{Optical setup}
		\subsection{Photoluminescence spectroscopy}
		\subsection{Absorption spectroscopy}
	\section{Sample characterization}
		\subsection{Electrical properties}
		\subsection{Narrow linewidth}
	\section{Gate sweep in magnetic field \# full control}
		\subsection{Doping regimes}
		\subsection{g-factors}

\begin{appendices}
\chapter{Lineshapes and fitting procedures}
	\section{Asymmetric Lorentzian}
	\section{Fano Resonance Lineshape}
\end{appendices}

\end{document}