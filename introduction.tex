\chapter{Introduction}
Ever since Andre Geim and Konstantin Novoselov were awarded the Nobel Prize for their groundbreaking work on graphene, two-dimensional (2\textsc{d}) materials have become a centerpiece of condensed matter physics \cite{novoselov_electric_2004}. After graphene, hundreds of other 2\textsc{d} materials were predicted, including transition metal dichalcogenides (\tmds\!). These 2\textsc{d}-semiconductors have attracted a lot of attention recently because of their unique valley physics, using the momentum of charge carriers as a pseudo-spin degree of freedom, that can be optically addressed \cite{wang_electronics_2012}. The potential applications have created a hype around \tmdg monolayers, that led to important milestones in the new field of ``valleytronics''. For example the valley hall effect \cite{mak_valley_2014} that is analogous to the spin hall effect and results a valley-polarized drift perpendicular to a current, when the sample is illuminated with circular polarized light. Only recently the manipulation of the valley-index of excitons with femto-second light pulses was shown \cite{langer_lightwave_2018}. This could potentially allow optoelectronic computations, much faster than classical silicon-based information processing. But \tmdg monolayers have gained attention not only because of their potential applications, but because the nature of these materials offers a unique model system to study physics in two dimensions. On the spectroscopy side, this is mainly due to the special properties of optically excited excitons in \tmdg monolayers. The confinement to two dimensions enables exciton formation well above room temperature and offers interesting applications as well as a system to study few- and many-body physics \cite{chernikov_exciton_2014}. Point defects can lead to formation of quantum dots, that have shown characteristics of single-photon emitters \cite{srivastava_optically_2015}. The high absorption efficiency of their direct band gap enables strong coupling and the creation of exciton polaritons---bosonic superpositions of excitons and photons \cite{liu_control_2017,zhang_photonic-crystal_2018}.

The physical processes governing their light-matter interaction have yet to be understood completely \cite{koperski_optical_2017}. The optical spectrum of \tmds exhibits many features, whose identification is necessary to get an understanding of the involved physical processes and to pave the way for future applications. 

While the spectrum is governed by the formation and recombination of excitons, especially tungsten-based \tmds show a rich ensemble of peaks, whose origin has so far been elusive. The focus of this thesis lies primarily on the material \wse\!. Spectroscopic analysis reveals signatures of both direct and indirect semiconductors. Direct excitons are known from other direct band gap semiconductor so are trions---excitons in a bound state with an additional free charge carrier. However, they also show features of so called momentum-indirect excitons that have net momentum and can only decay via phonon sidebands. They are associated with and indirect band gap.

One key to advance the research on \tmds is sample quality. Spectral features can vary greatly from sample to sample, that can exhibit different levels of intrinsic unintentional doping as well as strain and contamination from the fabrication process. Meaningful spectroscopic studies are therefore tethered to the fabrication of high quality samples, that show a ``pure'' spectrum, including spectral lines that show little inhomogeneous broadening, no defect driven features and are compensated for unintentional doping. A step towards this goal can be achieved by suspending \tmds in hexagonal boron nitride which offers an optimal dielectric environment to observe spectral lines close to the homogeneous linewidth \cite{dean_boron_2010,cadiz_excitonic_2017}. To compensate for intrinsic doping, samples have to be gate tuned with an applied voltage, that alters the charge carrier density inside the flake.

The goal of this thesis was to establish a process to fabricate samples in this manner. It utilizes the well established method of mechanical exfoliation for the production of \tmdg mono- and bilayer samples and \hbng substrate and capping layer. Using the novel fabrication technique of ``hot pick-up and stamping'' \hbn-\tmdg heterostructures are built and contacted to gold structures, that are fabricated with contact lithography. This enables photoluminescence (\pl) and differential reflection measurements that demonstrate the increased quality and gate-tunability.

The complete fabrication of a sample that can be used in experiment can now be completed in around three days. Of the samples created for this thesis, one combined mono- and bilayer and an additional bilayer sample of tungsten-diselenide showed both narrow lines of between 2--7 meV and are gate-tunable between neutrality and a negative doping regime. 

The thesis is divided into three sections. The first part summarizes the physical properties of \tmds that are relevant for optical studies. Then the fabrication and characterization of \hbn-encapsulated samples is described in detail, followed by experimental analysis of the reflection and \pl spectrum of \wse mono- and bilayers for different charge densities and magnetic fields.

%This is compared to similar measurements in \wse bilayers, 



