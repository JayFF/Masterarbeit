\chapter{Introduction}
Ever since Andre Geim and Konstantin Novoselov were awarded the Nobel prize for their discovery of graphene, two-dimensional materials have become a centerpiece of condensed matter physics\cite{novoselov_electric_2004}. After graphene, a lot of 2\textsc{d} materials were found, including transition metal dichalcogenides (\tmds\!). These 2\textsc{d}-semiconductors have attracted a lot of attention recently because of their unique valley physics, incorporating a pseudo-spin-like degree of freedom, that can be optically addressed\cite{wang_electronics_2012}. This has created a hype around \tmdg monolayers, that led to important milestones in the new field of ``valleytronics'', notably the valley hall effect\cite{mak_valley_2014} and the coherent manipulation of excitons with femto-second light pulses\cite{langer_lightwave_2018}. But \tmdg monolayers have gained attention not only because of their potential applications, but because the nature of these materials offers a unique model system to study physics in two dimensions. On the optics side, this is mainly due to the special properties of optically excited excitons in \tmdg monolayers. The confinement to two dimensions enables exciton formation well above room temperature and offers interesting applications as well as a system to study few- and many-body physics\cite{chernikov_exciton_2014}. Point defects can lead to formation of quantum dots, that have shown characteristics of single-photon emitters\cite{srivastava_optically_2015}. The high absorption efficiency of their direct band-gap enables strong coupling and the creation of exciton polaritons, bosonic superpositions of excitons and photons\cite{liu_control_2017,zhang_photonic-crystal_2018}.

However, the research into these materials is still young, and the physical processes governing their light-matter interaction are not completely understood\cite{koperski_optical_2017}. The optical spectrum of \tmds still shows a lot of unknown features, that need to be identified to get a good understanding of the involved physics and to pave the way for future applications. 

While the spectrum is governed by the formation and recombination of excitons, especially tungsten-based \tmds show a rich ensemble of different peaks. The focus of this thesis lies primarily on the material \wse\!. Its unique band structure puts it between direct and indirect semicondurctors and shows signatures of both types. Direct excitons and charged trions as well as momentum-indirect excitons, that decay via phonon sidebands. Indirect excitons offer a wide-ranging model to explain the complex spectrum of \wse\!. Yet the research on \tmds is still limited by sample quality. 

Spectral features can vary greatly from sample to sample, that can exhibit different levels of intrinsic unintentional doping, strain and contamination from the fabrication process. Meaningful spectroscopic studies are therefore tethered to the fabrication of high quality monolayers, that show a ``pure'' spectrum, including spectral lines that show little inhomogenious broadening, no defect driven features and are compensated for unintentional doping. A big step towards this goal can be achieved by suspending \tmds in hexagonal boron nitride, that offers an optimal dielectric environment to observe spectral lines, close to the homogenious linewidth\cite{dean_boron_2010,cadiz_excitonic_2017}. To compensate for intrinsic doping, samples have to be gate tuned with an applied voltage.

The goal of this thesis was to establish a pipeline to fabricate samples in this manner. This process utilizes the well established method of mechanical exfoliation for the production of \tmdg monolayer samples and \hbn substrate and capping layer. Using the novel fabrication technique of ``hot pick-up and stamping'' \hbn-\tmdg heterostructures are built and contacted to gold structures, that are fabricated with contact lithography. Photoluminescence and differential reflection measurements demonstrate the increased quality and gate-tunability.

The thesis is devided into three sections. The first party summarizes the physical properties of \tmds that are relevant for optical studies. Then the fabrication and characterization of \hbn-encapsulated samples is described in detail followed by experimental analysis of the reflection and photoluminescence spectrum of \wse monolayers for different charge densities and magnetic fields.

%This is compared to similar measurements in \wse bilayers, 



