\chapter{Physical properties of transition metal dichalcogenide monolayers}

\section{Crystal structure and symmetries}

Like to all layered materials, \tmds consist of large covalently bound sheets, that are held together by the weak van-der-Waals force. And similar to graphite these sheets have a hexagonal lattice structure and form layers of only one unit-cell in the out-of-plane axis. It is the details of these unit-cells that distinguish \tmds from graphite and other layered materials. The fundamental building block of the bulk crystal is a single sheet, the monolayer that consists of three atomic layers -- A layer of transition metal atoms like tungsten (\textsc{w}) or molybdenum (\textsc{M}{\footnotesize o}) sandwitched between two layers of chalcogen atoms like sulfur (\textsc{s}), selenium (\textsc{s}{\footnotesize e}) or tellurium (\textsc{t}). This thesis is primarily concerned with tungsten-based \tmds\-: \wse and \ws. \textsc{Tmd}s can be in different phases, that have a different crystal structure as well as different electronic properties. The semiconducting phase is called 2\textsc{h}. In this configuration every transition-metal atom has six neigbouring chalcogen atoms and forms a trigonal prismatic unit-cell, with the transition-metal in the center as depicted in Figure (Figure). A \tmdg monolayer exhibits a \textsc{d}$^1_{3h}$-symmetry. The unit-cell is invariant under 3-fold rotation as well as in-plane reflection. In the top-view (Figure) this looks similar to the hexagonal lattice structure of graphene, but with the key difference of a broken inversion symmetry. When the unit-cell is inverted with the transistion metal atom as its inversion center, the chalcogen atoms wind up in empty locations and so they to with any possible inversion point. 

\section{The valley zeeman effect}

Splitting of energy levels \cite{srivastava_valley_2015}