\section{Mechanical exfoliation}

There exist two major procedures to produce monolayers of layered materials like \textsc{tmd}'s. The bottom-up approach is called chemical vapor deposition (\textsc{cvd}). This method is the only feasible way for large scale production of \textsc{tmd}-monolayers. However, to obtain high quality single-crystaline samples with few impurities the top-down approach of mechanical exfoliation has so far proven to be supperior[References]. The key property of all layered materials is the contrast between the strong in-plane covalent bonds and the week interlayer van der Waals forces. As a result, these forces can be broken easily with something as basic as adhesive tape and mono-atomic layers of the order of 5 to 30 µm can be obtained. The process goes as follows[figure]: First, material is exoliated from bulk crystal. Then new stripes of tapes are used to further thin down the material on the last one. This is iterated a couple of times -- the exact number is a compromise between th probability to optain thin layers and the size of these layers. The last step is then to bring the tape in contact with the target substrate, i.e. silicon with thermal oxide. To maximize the amount of material, that is released from the tape the silicon wafers have to be prepared with oxygen plasma to ensure a clean surface. After the tape is brought in contact with the wafer, it is heated up to 90°C for two minutes. After cooling down the tape can be pulled of carefully and the residual material on the wafer can be examined with an optical microscope to find monolayers of a suitable size. For \textsc{tmd}'s monolayers can be easily spotted with the naked eye using a magnification of five times or more.

\subsection{Layer number}

To verify the thickness of a potential candidate flake several methods are available that vary both in accuracy and in effort. The easiest and surprizingly accurate method is to use the judgement of an experienced scientist. Both optical contrast and color distinct monolayers from bilayers and thicker flakes, so good regions can ideally be identified without using more advanced techniques. The next step is to qunatify contrast and color using a camera on the microscope and image analysis tools. This has been done extensively in [Victors Arbeit] and is still easier than other methods. However, to leverage the required fine tuning of image parameters for identification also the quality of the photographs has to be precise. Especially the lighting has to be as homogenious as possible and since the image analysis has to be adjusted to it, it should also remain constant in time. Both properties become more crucial with a reflective substrate like silicon. Under these conditions, the available microscope proved to be less reliable than on \textsc{pdms} or other transparent subtrates. That is why I had to rely on photoluminescence imaging, which 