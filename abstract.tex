\addchap*{Abstract}
%Legal disclaimer: This thesis contains more bullshit than the \textsc{eu-bsgvo} allows. Children, elderly people and pregrant women therefore are not allowed to read beyond this abstract and should consult a doctor if they experience the desire to facepalm for more than two consecutive days.\newline

Two dimentional semiconductors made of transition metal dichalchogenides (\tmds\!) have attracted a lot of interest because of their unique optical and electrial properties like a direct band gap, strong spin-orbit coupling, valley polarization and a high exciton binding energy. Optical spectroscopy of these materials has however been limited by inhomogneious broadening of spectral lines related to charge defects in the dielectric environment. Also, their varying degrees of intrinsic unintentional doping have made comprehensive studies of the reflection and photoluminescence impossible without external control over the density of free charge carriers. Both of these issues have been tackled during the course of this master's thesis. Using high quality \hbng as a substrate results in narrow linewith \pl\ spectra and the electric contacting to a lithographicly written gold-structure gives rise to gate-tunability. Confocal spectrocopy of a sample of \wse both in the neutral and negative regime at different magnetic fields helps to identify previously misunderstood features.\newline
\begin{center}
\par\rule[8pt]{0.7\textwidth}{1pt}
\end{center}
Die besonderen optischen und elektronischen Eigenschaften zweidimensionaler Halbleiter aus einzelnen Schichten von Übergangsmetall-dichalgogeniden (\tmds\!) machen diese Materialien zu interessanten Forschungsobjekten. Die Kombination aus einer direkten Bandlücke im sichtbaren Bereich sowie einer starke Spin-Bahn-Kopplung, Valley-Polarisation und einer hohe Exziton Bindungsenergie ist einzigartig. Allerdings wird die Untersuchung dieser Systeme durch optische Spektroskopie durch Ladungsdefekte in der dielektrischen Umgebung erschwert, da Spektrallinien dadurch inhomogen verbreitert werden. Auch die intrinsische Dotierung, die von Probe zu Probe variieren kann macht ein vollständiges Verständnis des Spektrums ohne externe Kontrolle der Ladungsdichte unmöglich. In dieser Arbeit werden diese beiden Aspekte angegriffen: Um eine defektfreie dielektrische Umgebung zu schaffen, werden die Proben in hexagonales Bornitrid eingebettet und mit lithographischen Goldstrukturen kontaktiert um die Ladungsdichte durch den Feldeffekt zu steuern. In konfokaler Spektroskopie wird das neutrale und negativ geladene Spektrum einer \wse\!-Probe untersucht und das Verhalten der Spektrallinien im Magnetfeld beobachtet um bisher unverstandene Lininen zu identifizieren.